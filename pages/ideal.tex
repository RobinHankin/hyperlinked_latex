\newcommand{\idealhelptext}{
  \section{Ideal help here!}

  An {\bf ideal} is a \subspace\ of a \liealgebra\ that is closed
  under Lie brackets.

\mainbox{Formally, if $L$ is a \liealgebra\, then $I$ is an ideal if}{
  {x\in L, y\in I\longrightarrow\bracket{x}{y}\in I}}

Alternatively, we might write $x\in L\longrightarrow [I,x]\in I$.
Ideals are analogous to normal subgroups in group theory.

\subsection*{Notes}
\begin{itemize}
  \item The condition of being an ideal is stronger than the condition
    of being a \subalgebra.  For $I$ to be a subalgebra of $L$, it
    only needs the property that $\bracket{I}{x}\in I$ for any $x\in
    I$; it is not necessary that $\bracket{I}{x}\in I$ for any $x\in
    L$.
\end{itemize}



\subsection*{Examples}


\subsection*{Facts}
\begin{itemize}
\item  Suppose $I$ and $J$ are ideals of $L$.  Then we can construct new
ideals in different ways:
\begin{itemize}
\item The intersection $I\cap J$ is an ideal of $L$.
\item $I+J=\left\{x+y\colon x\in I,y\in J\right\}$ is an ideal of $L$.
\item $\bracket{I}{J}=\operatorname{Span}(\bracket{x}{y}\colon x\in
  I,y\in J)$ is an ideal of $L$.  Note that this is automatically
  a \subspace\ of $L$.  Note that $\bracket{L}{L}=L'$ is the
  \derivedalgebra\ of $L$.
\end{itemize}
\end{itemize}

 \newpage
}

\newcommand{\representationhelptext}{
  \section{Representation help}

A {\bf representation} of $L$ is a \liealgebrahomo

\mainbox{Formally, if $L$ is a \liealgebra\ over a field $\mathbb{F}$
  and $V$ a finite dimensional \vectorspace\ over $\mathbb{F}$, then a
  representation of $L$ is a \liealgebrahomo:}
        {\liealgebrahomomath\colon L\longrightarrow\glv}

Loosely, we sometimes say that $V$ is ``a representation of $L$'' with
$\phi$ understood.  Then we can specify a representation by giving a
\liealgebrahomo, specifically $\liealgebrahomomath\colon
L\longrightarrow\mathsf{gl}(n,\mathbb{F})$ in which case we call
$\phi$ a matrix representation.

A representation is equivalent to a \module.

If the map \liealgebrahomomath\ is an isomorphism, then we say that
the representation is {\bf faithful}.


\subsection*{Examples}

\begin{itemize}
\item The \adjoint\ map is a representation of $L$.  We have
  $\adjointmath\colon L\longrightarrow\operatorname{\mathsf{gl}}(L)$,
  defined by $(\adjointmath x) y=\bracket{x}{y}$ which is a
  \liealgebrahomo.  Thus \adjointmath\ is a representation with $V=L$.
  The kernel of \adjointmath\ is the \centre\ of $L$: so if the centre
  of $L$ is trivial, the adjoint representation is faithful.
\item If $L$ is a Lie \subalgebra\ of $\operatorname{\mathsf{sl}}(V)$,
  then the inclusion map
  $L\longrightarrow\operatorname{mathsf{gl}}(V)$ is a \liealgebrahomo.
  This is the {\bf natural representation} of $L$.  The natural
  representation is always faithful.
\item 
\item 
\end{itemize}

 \newpage
}

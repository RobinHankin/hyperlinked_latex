\newcommand{\fieldhelptext}{
  \section{Field help,  right here}

  A {\bf field} is a set $F$ on which addition and multiplication is
  defined.  The archetype of a field is the set of real numbers (or
  the rationals) together with conventional addition and
  multiplication.

However, integers modulo~$p$ (where~$p$ is a prime number) form a
field as well.

\subsection{Examples}

The best example is the real numbers $\mathbb{R}$ together with
conventional addition and multiplication; formally one would write
$\left(\mathbb{R},+,\times\right)$.  Other examples would include the
rational numbers $\mathbb{Q}$, the complex numbers $\mathbb{C}$.

There are finite fields too, for example $\mathbb{Z}_p$, the integers
modulo (prime) $p$.



\subsection{Nonexamples}

The set of $n\times n$ matrices with real entries $M_n(\mathbb{R})$ is
not a field (if $n>1$) because there are many nonzero matrices that
cannot be inverted (any nonzero matrix with zero determinant).

\newpage
}

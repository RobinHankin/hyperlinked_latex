\newcommand{\modulehelptext}{
  \section{Module help}

A {\bf Lie module} is an alternative way of thinking about Lie algebra
\representation s.

\mainbox{Formally, if $L$ is a Lie algebra over $\mathbb{F}$, then a
  {\bf L-module} is a finite dimensional \vectorspace\ over
  $\mathbb{F}$, together with a map}
        {L\times V\longrightarrow V\qquad \left(x,v\right)\mapsto x\moduledot v}{
          where
          \begin{itemize}
          \item $(\lambda x+\mu y)\moduledot v=\lambda(x\moduledot v)+\mu(y\moduledot v
            )$
          \item $x\cdot(\lambda v+\mu w)=\lambda(x\cdot v)+\mu(x\cdot w)$
          \item $\bracket{x}{y}\cdot v=x\cdot(y\cdot v)-y\cdot(x\cdot v)$
          \end{itemize}
          whenever $x,y\in L$, $u,v\in V$, and $\lambda,\mu\in\mathbb{F}$.
        }

\subsection*{Examples}

\begin{itemize}
\item Given a \representation\ $\liealgebrahomomath\colon
  L\longrightarrow\glv$, we can consider $V$ to be an $L$-module by
  defining $x\cdot v$ as $\liealgebrahomomath(x)(v)$ whenever $x\in
  L$, $v\in V$.  The first two conditions are straightforward.  For
  the third, we have
  \[
  \bracket{x}{y}\cdot
  v=\liealgebrahomomath\left(\bracket{x}{y}\right)(v)=\bracket{\liealgebrahomomath(x)}{\liealgebrahomomath{(y)}}(v)
  \]
  and because the Lie bracket in \glv\ is just the commutator of
  linear maps, this is equal to
  \[
  \left(  \liealgebrahomomath(x)\liealgebrahomomath(y)-\liealgebrahomomath(y)\liealgebrahomomath(x)\right)(v)=
  \liealgebrahomomath(x)\liealgebrahomomath(y)(v)-\liealgebrahomomath(y)\liealgebrahomomath(x)(v)
  \] which by
  definition is $x\cdot(y\cdot v)-y\cdot(x\cdot v)$.
\item  If $I$ is an \ideal\ of Lie algebra $L$.  The FACTOR MODULE $L/I$ becomes an $L$-module by defining
  \[
  x\moduledot\left(I+y\right)=I + \left(\adjointmath x\right)(y) = I+\bracket{x}{y}
  \]
  Alternatively, we might say that the \quotient\ algebra $L/I$ is a
  Lie algebra with bracket defined as
  $\bracket{I+x}{I+y}=I+\bracket{x}{y}$.  So, regarded as an
  $L/I$-module, the Factor Module $L/I$ is actually the
  \adjoint\ \representation\ of $L/I$ on itself.
\item 
\end{itemize}

 \newpage
}

\newcommand{\adjointhelptext}{   % \adjoint and \adjointmath link here
\section{Adjoint help here}

The {\bf adjoint}, or {\bf adjoint homomorphism} is a map from a
\liealgebra\ $L$ to $\mathsf{gl}(L)$, the set of all linear maps from
$L$ to itself.  

\mainbox{Formally,
}{
  \operatorname{ad}\colon L\mapsto\mathsf{gl}(L); \qquad
(\operatorname{ad} x)(y)=\bracket{x}{y}
}

\subsection*{Notes}

The definition requires some unpacking.  If $x\in L$, then
$\operatorname{ad}(\cdot)$ maps $x$ to a member of $\mathsf{gl}(L)$.
We might write ``$\operatorname{ad}(x)=f$'' and then, noting that
$f\in\mathsf{gl}(L)$, ask what $f$ maps $y\in L$ to; the answer being
that $f(y)=\bracket{x}{y}$.  Alternatively, we might write

  \boxeq{\operatorname{(ad} x)(y)=\bracket{x}{y}}


For $\bracket{x}{\bracket{x}{\bracket{x}{y}}}$ we would have
$\left(\adjointmath x\right)^4(y)$.
 

  \subsection*{Facts}
  \begin{itemize}
  \item We can view the \adjoint\ as a map from a Lie algebra $L$ to
    the endomorphism group $\operatorname{End}(L)$ of $L$, given by
    $x\mapsto (\adjointmath x)$.  This map is a \liealgebrahomo, for
    \[
    \bracket{x}{y}\mapsto(\adjointmath \bracket{x}{y}) = (\adjointmath
    x)(\adjointmath y)-(\adjointmath y)(\adjointmath
    x)=\bracket{\adjointmath x}{\adjointmath y}\] (the proof uses the
    Jacobi identity; it is not obvious).
  \item The adjoint map is a \derivation, for $(\adjointmath
    x)\bracket{y}{z} = \bracket{(\adjointmath x) y}{z}
    +\bracket{y}{(\adjointmath x) z}$.
    \item The adjoint map is a \representation.

  \end{itemize}
  
  
 \newpage
}

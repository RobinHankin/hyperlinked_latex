\newcommand{\squarematrixhelptext}{
  \section{Square matrices}

  A {\bf square matrix} is a matrix that is square, that is, it has
  the same number of rows and columns.



\subsection*{Examples}

The simplest non-trivial example would be a $2\times 2$ matrix:


\[ \begin{bmatrix}  a&b\\ c&d\end{bmatrix} \]

  The general case has $n$ rows and $n$ columns:

\[
\begin{bmatrix}
    x_{11} & x_{12}  & \dots  & x_{1n} \\
    x_{21} & x_{22}  & \dots  & x_{2n} \\
    \vdots & \vdots  & \ddots & \vdots \\
    x_{n1} & x_{n2}  & \dots  & x_{nn}
\end{bmatrix}
\]

\subsection*{Facts}
\begin{itemize}
  \item A square matrix may be multiplied by itself.
  \item A matrix has a \trace\ if and only if it is square.
  \item The transpose of a square matrix is square.
  \item You can take the \matexp\ of a square matrix.
  \item In statistics, a variance-covariance matrix is square
  \item A square matrix has eigenvalues and eigenvectors.
\end{itemize}

 \newpage
}

\newcommand{\simplehelptext}{
  \section{Simple help here!}

A Lie algebra $L$ is {\bf simple} if it is not abelian, and has no
\ideal s other than \boldzero\ and $L$ itself.  Compare \semisimple.


\subsection*{Notes}

The requirement for $L$ to be abelian only affects one-dimensional Lie
algebras.  Without this requirement, the one-dimensional Lie algebra
would be simple (because it has no nontrivial ideals) but not
semisimple (because it has a \solvable\ \ideal---namely, itself) which
is nuts.


\subsection*{Facts}
\begin{itemize}
\item \slncla is a simple Lie algebra if $n\geq 2$.
  \item Every \semisimple\ Lie algebra is a \directsum\ of \simple\ Lie algebras.
\end{itemize}
 \newpage
}

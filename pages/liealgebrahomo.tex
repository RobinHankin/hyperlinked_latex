\newcommand{\liealgebrahomohelptext}{
  \section{Lie algebra homomorphism help here!}

  (this page is about Lie algebra homomorphisms).  A {\bf
    homomorphism} (sometimes {\bf Lie algebra homomorphism}) is a
  linear map that preserves Lie brackets.  Note that a {\bf Lie
    homomorphism} is a different thing (which refers to \module s).

  \mainbox{Formally, if $L_1$ is a Lie algebra, then $\phi\colon
    L_1\longrightarrow L_2$ is a Lie algebra homomorphism
    if}{\phi(\alpha x+\beta y)=\alpha\phi(x)+\beta\phi(y)\qquad\mbox{and}\qquad
    \phi\left(\bracket{x}{y}\right)=\bracket{\phi(x)}{\phi(y)}}

  Alternatively, we might require that
  \begin{itemize}
  \item $\phi$ is linear: $\phi(\alpha x+\beta y)=\alpha\phi(x)+\beta\phi(y)$
  \item $\phi$ commutes past Lie brackets, that is:
    $\phi\left(\bracket{x}{y}\right)=\bracket{\phi(x)}{\phi(y)}$
  \end{itemize}
  
  \subsection{Examples}
  \begin{itemize}
    \item A good homomorphism to consider is furnished by an
      \ideal\ $I$ of $L$.  Define $\phi$ as the map that takes an
      element $x\in L$ to its coset $I+x$ (this is the
      \quotient\ map).  Formally, $\phi\colon L\longrightarrow
      L\quotientmath I$ with $\phi(x)=I+x$.  See the {\bf three
        isomorphism theorems} below.
    \item Consider the \trace\ map $\tracemath\colon
      M_n(\mathbb{F})\longrightarrow\mathbb{F}$ which takes an
      $n\times n$ matrix with elements in $\mathbb{F}$ to its \trace.
      This is a Lie algebra homomorphism as both sides of the
      definition are zero:
      $\tracemath(\bracket{x}{y})=\tracemath(xy=yx)=0$; but
      $\bracket{\tracemath(x)}{\tracemath(y)}=0$ as well, for the
      bracket is taken in the abelian Lie algebra $\mathbb{F}$.  The
      kernel of the \trace\ map is matrices of trace zero,
      $\mathsf{sl}(n,\mathbb{F})$.  Also note
           \end{itemize}
  
  \subsection*{Isomorphism theorems}
  
  There are three isomorphism theorems for Lie algebra homomorphisms
  that are analogous to the three isomorphism theorems in group
  theory:
  \begin{itemize}
  \item If $\phi\colon L_1\longrightarrow L_2$ is a Lie algebra
    homomorphism, then $\operatorname{ker}\phi$ is an \ideal\ of
    $L_1$; also, the image of $\phi$, $\operatorname{im}\phi$ is a
    \subalgebra\ of $L_2$; and finally the \quotient\ algebra
    $L_1\quotientmath\operatorname{ker}\phi\cong\operatorname{im}\phi$.
  \item If $I$ and $J$ are \ideal s of a Lie algebra, then
    $(I+J)\quotientmath J\cong I\quotientmath(I\cap J)$.
  \item If $I$ and $J$ are \ideal s of a Lie algebra $L$, and
    $I\subseteq J$, then $J\quotientmath I$ is an ideal of
    $L\quotientmath I$, and furthermore $(L\quotientmath
    I)\quotientmath (J\quotientmath I)\cong L\quotientmath J$.
  \end{itemize}
  
  \newpage
}
